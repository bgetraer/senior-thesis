% Senior thesis or Junior Project Abstract -----------------------------------------------------

% The premise for this paper is that there are significant unpredicted melt anomalies within the signal of the Greenland ice sheet seen in the GRACE timeseries. I believe that the global  spherical harmonic basis in which the GRACE data are released bias local Greenland signal with respect to other global gravitational anomalies, and obscure significant spatial information about the location and scale of sub-seasonal melt events. By transforming the GRACE time-series into bases which have concentrated support over Greenland we can gain a more accurate and informative understanding of where large scale mass-wasting events are occurring on the ice sheet, which in turn can help to confirm and inform our developing understanding of ice sheet -- atmosphere interactions.

Melting ice from the Greenland Ice-Sheet has accounted for an increasing percentage --- now estimated at $25\%$ --- of rising global mean sea-level since the early 1990s. As recently as 2016, gravimetric and altimetric studies of Greenland melting rates found  increasing rates of ice loss, which have not been borne out in GRACE gravimetric observations over the last few years (2015--2017). We hypothesize that the true trend of Greenland ice loss between 2003--2017 is linear, and that deviations from the linear trend may be explained by inter-annual variability in climate. We demonstrate a novel application of 2-dimensional discrete wavelet analysis to the GRACE dataset to recover spatial structure of inter-annual variability in ice loss, focusing on the unusual melt and accumulation seasons of 2012--2014. Finally, we compare our interpretation of the 2012--2014 anomaly in spatial scale and location to the results of others using independent atmospheric, altimetric, and meteorologic data sources. \\[3em]

\textbf{Key Points:}
\begin{enumerate}
	\item We focus on inter-annual variability of the Greenland ice loss trend.
	\item We analyze subregional signals using discrete wavelet transforms.
%	\item We find unexpected periodic structure of $3$--$7$ years in the Greenland ice loss trend.
	\item We define the 2012--2014 anomaly in spatial structure.

\end{enumerate}