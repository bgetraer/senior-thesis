% Senior thesis or Junior Project Abstract -----------------------------------------------------

% The premise for this paper is that there are significant unpredicted melt anomalies within the signal of the Greenland ice sheet seen in the GRACE timeseries. I believe that the global  spherical harmonic basis in which the GRACE data are released bias local Greenland signal with respect to other global gravitational anomalies, and obscure significant spatial information about the location and scale of sub-seasonal melt events. By transforming the GRACE time-series into bases which have concentrated support over Greenland we can gain a more accurate and informative understanding of where large scale mass-wasting events are occurring on the ice sheet, which in turn can help to confirm and inform our developing understanding of ice sheet -- atmosphere interactions.

Melting ice from the Greenland Ice-Sheet has accounted for an increasing percentage---now estimated at $25\%$---of rising global mean sea-level since the early 1990s. As recently as 2016, gravimetric and altimetric studies of Greenland melting rates found increasing rates of ice loss, which have not been borne out in GRACE gravimetric observations over the last few years (2015--2017). I investigate the correlations of atmospheric variables from MERRA-2 climate model reanalysis to show the ways in which temperature over the Greenland Ice Sheet has changed over the MERRA-2 (1980--) and GRACE (2003--2017) records. Our results not only confirm that temporal and spatial changes in GRACE derived mass loss are coincident with changes in near surface temperature, but demonstrate some of the limitations in GRACE spatial resolution, and contextualize recent variability in ice loss within the variability and long term trend of Greenland temperature. As Greenland Ice Sheet melting continues to be more unpredictable than early GRACE studies predicted, context is extremely important in both interpreting and communicating trends in ice loss. \\[3em]

\textbf{Key Points:}
\begin{enumerate}
	\item I focus on inter-annual variability of the Greenland ice loss trend.
	\item I contextualize recent variation in melt and temperature through analysis of 1980--2017 MERRA-2 climate reanalysis data. 
%	\item We find unexpected periodic structure of $3$--$7$ years in the Greenland ice loss trend.
%	\item .

\end{enumerate}