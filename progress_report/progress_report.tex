\documentclass[11pt]{report}
\clubpenalty=10000
\widowpenalty=10000

% input packages that I call bc they are in the template and I might want them.
\input{/Users/benjamingetraer/Documents/IndependentWork/LatexFiles/packages}

% It is handy to define new commands for text that occurs frequently (see Discussion)
\newcommand{\MT}{^{\mathrm{MT}}}
\newcommand{\ga}{\gtrsim}
\newcommand{\Lpot}{(L+1)^2}
\newcommand{\WS}{^{\mathrm{WS}}}

%--Format the section headers

% FJS Changed this... I didn't like the numbering or the
% indentation... so I introduced a fake chapter Main Text. 
\setcounter{secnumdepth}{0}
\setcounter{tocdepth}{5}

%--set the page formatting--
\geometry{hmargin={1.6in,1.1in},vmargin={1.5in,1.2in}}
\doublespacing

%***********************************************************************************************************************************************************************************************************************************************************************************************************************************************************************

\begin{document}
%--front matter needs roman pagination--
\pagenumbering{roman}

%--Title Page--
\thispagestyle{empty}
  \begin{center}
    \textsc{\LARGE Comparison of GRACE Derived Greenland Mass Wasting to Physical Drivers of Ice Loss From MERRA-2 Reanalysis} 
  \end{center}
  \vspace{.6in}
  \begin{center}
    Benjamin Getraer 
  \end{center}
  \vspace{.6in}
  \begin{center}
    \textsc{Senior Thesis \\ 
    Presented to the Faculty \\
    of Princeton University \\
    in Candidacy for the Degree \\
    of Bachelor of Arts}
  \end{center}
  \vspace{.3in}
  \begin{center}
    \textsc{Recommended for Acceptance \\
    by the Department of \\
    Geosciences \\}
    Adviser: Laure Resplandy \\
    Second Reader: Frederik J.~Simons \\
  \end{center}
  \vspace{.3in}
  \begin{center}
\today
  \end{center}
  
  \clearpage

%--Copyright Page--
\thispagestyle{empty}
\vspace*{3in}
\begin{center}
\emph{This paper represents my own work in accordance with University regulations,} \\
Benjamin Getraer %%Sign here
\end{center}
\clearpage

%--Abstract--  
\addcontentsline{toc}{chapter}{Abstract}
\begin{center}
\Large \textbf{Abstract}
\end{center}
 % Senior thesis or Junior Project Abstract -----------------------------------------------------

% The premise for this paper is that there are significant unpredicted melt anomalies within the signal of the Greenland ice sheet seen in the GRACE timeseries. I believe that the global  spherical harmonic basis in which the GRACE data are released bias local Greenland signal with respect to other global gravitational anomalies, and obscure significant spatial information about the location and scale of sub-seasonal melt events. By transforming the GRACE time-series into bases which have concentrated support over Greenland we can gain a more accurate and informative understanding of where large scale mass-wasting events are occurring on the ice sheet, which in turn can help to confirm and inform our developing understanding of ice sheet -- atmosphere interactions.

Melting ice from the Greenland Ice-Sheet has accounted for an increasing percentage---now estimated at $25\%$---of rising global mean sea-level since the early 1990s. As recently as 2016, gravimetric and altimetric studies of Greenland melting rates found increasing rates of ice loss, which have not been borne out in GRACE gravimetric observations over the last few years (2015--2017). I investigate the correlations of atmospheric variables from MERRA-2 climate model reanalysis to show the ways in which temperature over the Greenland Ice Sheet has changed over the MERRA-2 (1980--) and GRACE (2003--2017) records. Our results not only confirm that temporal and spatial changes in GRACE derived mass loss are coincident with changes in near surface temperature, but demonstrate some of the limitations in GRACE spatial resolution, and contextualize recent variability in ice loss within the variability and long term trend of Greenland temperature. As Greenland Ice Sheet melting continues to be more unpredictable than early GRACE studies predicted, context is extremely important in both interpreting and communicating trends in ice loss. \\[3em]

\textbf{Key Points:}
\begin{enumerate}
	\item I focus on inter-annual variability of the Greenland ice loss trend.
	\item I contextualize recent variation in melt and temperature through analysis of 1980--2017 MERRA-2 climate reanalysis data. 
%	\item We find unexpected periodic structure of $3$--$7$ years in the Greenland ice loss trend.
%	\item .

\end{enumerate}
 \clearpage

%--Acknowledgements--  
\addcontentsline{toc}{chapter}{Acknowledgements}
\begin{center}
\Large \textbf{Acknowledgements}
\end{center}
% Senior thesis or Junior Project Acknowledgements  -----------------------------------------------------

Thank you to my thesis adviser Prof.~Laure Resplandy who has helped me understand and think through atmospheric processes and the direction this project has taken this year.
Thank you to my Junior Paper adviser Prof.~Frederik J.~Simons who helped me with
the conceptualization, direction, and revision of earlier portions of this project. Thank you to the second reader of my Fall Junior Paper, Prof.~Jessica Irving for feedback, suggestions, and encouragement. Thanks to Dr.~Amanda
Irwin Wilkins and ``The Hare'' writing workshop group for discussing
and editing various figures and drafts of my Junior Papers. Thank you to Prof.~Adam C.~Maloof who taught me \LaTeX, and to Dr.~Chris Harig for providing some of the data files. Thank you to Prof.~Gabriel Vecchi for insight into atmospheric and oceanic processes. Thank you to Jean Getraer, Andrew Getraer, Jonathan Feld, Rae Perez, and Zach Smart for proof-reading various Junior Paper drafts.  Lastly, thank you to the numerous professors and graduate and undergraduate students in the Princeton Department of Geosciences for feedback, encouragement, and constructive criticism on various presentations of preliminary results. A big thanks especially to all of the people who asked me, ``What is your independent work about?'' and then patiently listened while I struggled to articulate this project in a way that made any sense.

\clearpage

%--Table of Contents--  
\thispagestyle{empty}
\tableofcontents
\clearpage

\listoffigures
\listoftables
\clearpage

%--Set up fancy header-- 
\fancyhead{}
\fancyfoot{}
\pagestyle{fancyplain}
\rhead{\fancyplain{\thepage}{\noindent \textsc{\rightmark} \hfill \thepage~of~\pageref{LastPage}}}
\rfoot{\hrule \today \hfill Benjamin Getraer}
\pagenumbering{arabic}

%--Reset the page numbers and set them to arabic-- 
{\newpage\renewcommand{\thepage}{\arabic{page}}\setcounter{page}{1}}

%--Have sections but use chapter counters
\addcontentsline{toc}{chapter}{Main Text}

\section{Introduction \label{sec:introduction}}

Average global surface temperature is rising at an increasing rate --- approximately $0.09\degree$~C per decade since 1880, and approximately $0.26\degree$~C per decade since 1979 \cite[][]{ipcc2013_atmosphere}
--- with the past five years (2014--2018) being approximately $0.84\degree$~C warmer than the 1880--2018 average \cite[][]{NOAA_stateofclimate}. Earth's warming climate has contributed to significant melting of the Greenland ice sheet, and recent ice loss is estimated at $-244$~Gt per
year \cite[][]{Harig+2015a,Harig+2016}. The Greenland Ice Sheet covers just over 1\% of
Earth's surface, and, if completely melted, would raise sea level by over $7$~m
\cite[][]{ipcc2013_cryosphere}.
Global melting of ice sheets and glaciers accounts for almost half of recently observed rising sea levels, with Greenland alone contributing almost $25$\% of the $3$~mm per year of sea level rise observed since 1993
\cite[][]{Nerem}.  Our broad goal is to understand the nature and cause of deviations from modeled rates of Greenland ice melt in order to better understand, predict, and communicate the changing conditions of the planet.


%\begin{figure}[h!]
%\centering
%\makebox[\textwidth][c]{\includegraphics[height=0.37\textheight]
%{Figures/HarigGetraerTrend.pdf}}
%\caption[Greenland Mass Trend: 2003--2017]{Total mass changes for Greenland over the complete GRACE record using equivalent methods to \cite{Harig+2016}. Shown in \textbf{A} are the $\underline{m}_1$ (linear) and $\underline{m}_2$ (quadratic) models for 01/2003--06/2013, comparable to previous estimates of the mass trend \citep{Harig+2016}. Note the significant departure of the extrapolated $\underline{m}_2$ model from the continuing signal. Shown in \textbf{B} is the $\underline{m}_1$ linear model for 01/2003--06/2017 with the standard of deviation of its residuals. Note that the $\underline{m}_1$ model does not significantly change after including the entire GRACE record. Error bars represent $2\sigma$ based on the combined variance of modeled Slepian coefficients $f_{\alpha}$ (see \cite{Harig+2016}, as well as \cite{getraerFall,getraerSpring}). This figure appeared in \cite{getraerFall,getraerSpring}, here with minor updates.} \label{fig:Getraer}
%\end{figure}

Ice loss on the Greenland Ice Sheet has been observed in
gravitational measurements from NASA's Gravity
Recovery and Climate Experiment (GRACE), satellite and airplane based
altimetry, and energy balance models, finding acceleration of melt in the ice mass signal over most of the last two decades \cite[][]{khan2015,Harig+2016}. Rates of ice loss increase by a
combination of greater discharge from calving glacier termini at the edges of
the ice-sheet and decreased surface mass-balance, the difference between
seasonal snow accumulation and melting \cite[][]{khan2015,enderlin2014}.
Significant inter-annual variability and asynchronicity has been observed in
the discharge rates of the Greenland Ice Sheet's major drainage basins, while
surface mass-balance is comparatively more predictable
\cite[][]{mcmillan2016,enderlin2014}. Both contributions to ice loss accelerated between 2000--2012, combining for a total acceleration of ice mass
estimated at nearly $-30$~Gt per year$^2$ over all of Greenland
\cite[][]{velicogna2009,enderlin2014,bevis2019}.

%\begin{figure}[h]
%\centering
%\makebox[\textwidth][c]{\includegraphics[height=0.5\textheight]
%{Figures/jetstream.pdf}}
%\caption[Atmospheric Circulation Around Greenland]{Example of atmospheric conditions at the $250$~hPa isobar over the North Atlantic preceding record Greenland Ice Sheet surface melt, 07/09/2012 (from MERRA-2 reanalyzed data). Isobar height contours are labeled in $10^4$~m, and wind vectors are shown by arrows. Note the location of the northern polar jet stream, dividing the low and high isobar heights around the $1.05\times 10^{4}$~m contour. The temporary North Atlantic Rossby wave is labeled as the ``HI'' pressure anti-cyclone moving north towards southern Greenland, with a complementary ``LO'' pressure cyclone centered over Labrador. Note that the jet stream is deflected through the Labrador Sea and Baffin Bay, along the West Coast of Greenland. As a result of these conditions, the ``LO'' pressure over Reykjav\'ik is pushed north, and the pressure difference between Reykjav\'ik  and Azores is lowered, resulting in an negative NAO index (top left).} \label{fig:jetstream}
%\end{figure}

A study by \cite{Harig+2016} modeling the mass of the Greenland Ice Sheet
using GRACE data products showed deviations from the long-term accelerating
trend, starting with a high level of melt in the summer of 2012, and followed by two
summers of little melting in 2013 and 2014 \cite[see
Figure~\ref{fig:Getraer} A, comparable to][their
Figure~4]{Harig+2016}. Our analysis of the complete GRACE data set (2002--2017) using identical methods showed a linear,
not accelerating, trend of ice loss for the Greenland Ice Sheet, constraining the observed unexpected deviations to an unusually large melt summer of 2012 followed by a summer of unusually little melt in 2013 (see
Figure~\ref{fig:Getraer} B). 


The anomalous seasons of 2012--2013 have received attention in recent literature by studies attempting to understand how surface mass balance processes produce such inter-annual variability. Correlations have been found with climate indices such as the phase of the North Atlantic Oscillation (NAO) \cite[][]{mcmillan2016,bevis2019,getraerFall}, transient atmospheric transport of warm air and water vapor in so-called "atmospheric rivers" \citep{mattingly2018}, and non-radiative energy flux enhanced by short-term cloud cover \citep{solomon2017}. 

%\begin{wrapfigure}{r}{0.5\textwidth} 
%\includegraphics[width=1.1\linewidth]{Figures/thegrid.pdf}
%\caption[The Discrete Grid Around Greenland]{A grid is defined in the global basis on a face of the Cubed Sphere centered on Greenland, upon which the gravitational anomaly is evaluated from the GRACE spherical harmonic solutions. In the image basis the grid is cartesian with length $256$. Grid lines in the image basis represent the diminishing spatial support of wavelets of different levels, from $\zeta=8$ (the entire image) to $\zeta=1$ (a unit grid cell). Note that in reality, each wavelet level has coverage over the entire image. The dotted line around Greenland is a coastal buffer of $0.5^{\circ}$ as in \cite{Harig+2016}. This Figure appeared in my Spring JP.} \label{fig:thegrid}
%\end{wrapfigure}



%% SIGNIFICANCE OF THE POLAR JET STREAM
\subsection{Atmospheric Circulation Over the Greenland Ice Sheet}
The atmospheric circulation affecting the Greenland Ice Sheet is broadly controlled by the position of the polar jet stream in the northern hemisphere \cite[][]{hanna2013,mattingly2018}. The northern polar jet stream is a strong current of air moving generally eastward, dividing the warm, high pressure air of the temperate mid-latitudes and the cold, low pressure air of the Arctic. In the North Atlantic, the average position of the polar jet results in a low pressure Arctic system centered near Iceland, and a high pressure temperate system centered near the Azores. The strength of the polar jet stream moving air zonally across the North Atlantic is determined by the relative meridional pressure difference across the North Atlantic, and varies irregularly in what is called the North Atlantic Oscillation (NAO).

The NAO is often indexed for use in climate analyses by differencing the atmospheric pressure over the North Atlantic and a "loading pattern" of pressure determined by principal component analysis to capture the average meridional pattern \cite[][]{hurrelNAO,cpcNAO}. A positive phase NAO index reflects a stronger meridional difference in pressure than usual, while negative values reflect a weaker meridional difference than usual \cite[][]{hurrelNAO}. The index is a unitless ratio normalized by the variance of atmospheric pressure from the loading pattern.


In studying melt events on the Greenland Ice Sheet, many studies have used an NAO index to relate melting conditions to atmospheric pressure patterns. Strong summer melt events often occur with a negative NAO index, suggesting that mechanisms which drive the NAO may have some kind of predictable implications for melting on Greenland \cite[][]{hanna2013,mattingly2018,mcmillan2016,bevis2019,getraerFall,hahn2018}. More precisely, the use of the NAO index suggests that meridional pressure differences which drive westerly winds across the North Atlantic are a physical mechanism for strengthening melt conditions over Greenland --- warm, moist air, and increased cloud cover \cite[][]{hanna2013,mattingly2018,mcmillan2016,bevis2019,getraerFall,hahn2018}. This assumption, however, obscures the fact that melt events are often caused by southerly winds advecting atmospheric rivers of warm, moist air north, which is driven by a zonal difference in pressure \cite[][]{hanna2013,mattingly2018}.

The polar jet does not maintain stable zonal flow, and regularly develops wiggles in which flow is diverted meridionally in large waves. During these events, known as a Rossby waves, temporary high-pressure systems push northward into the arctic accompanied by complementary low-pressure cyclones which develop on either side of the high-pressure block. The combined flow from these pressure systems advects warm air from the temperate mid-latitudes into the Arctic until the Rossby wave ``breaks'' and the jet stream return to its typical location.

Melt events on Greenland are generally driven by zonal differences in pressure resulting from Rossby wave systems with high pressure ``blocking'' over southern Greenland, which create atmospheric rivers advecting warm moist air over the ice sheet \cite[][]{mattingly2018}. In contrast, the NAO index measures against a meridional patterns of pressure, with the result that although melt events often correspond with a weaker NAO, a lower the NAO index does not necessarily imply a greater melt event.

High pressure blocking over southern Greenland is often correlated with a negative NAO index, because the northern excursion of high pressure systems influences meridional pressure balance. However, despite their correlation, the two are fundamentally different atmospheric patterns which drive advection of air in different directions. In some recent studies, the Greenland Blocking Index (GBI) has been used, which is essentially the average pressure over Greenland. While this is more specific and more relevant than the NAO index, it still does not directly reflect the meridional advection which drives melt.

In this paper, I explore simple indices which directly compare zonal pressure differences which flow of warm air into the arctic over the Greenland Ice Sheet, and compare those results to the NAO and GBI indices.



 
 
% These atmospheric conditions are illustrated in Figure~\ref{fig:jetstream}, illustrating the Greenland blocking event in July 2012, immediately preceding record surface melting on the Greenland Ice Sheet on 07/11/2012 \cite[][]{hanna2013,mattingly2018}.



%The conclusion of many studies would be to suppose that the NAO is driving Greenland melt, and to look to patterns in the NAO index to understand the future of Greenland's ice sheet. This is highly concerning, as there are strong signs that the NAO has been trending more and more negative in recent years. However, it is crucial to note that there are significant theoretical obstacles in the correlation of the NAO index to melt. 



%The general conclusions drawn by these studies point to common atmospheric processes of uncommon duration or intensity. 





\subsection{Previous Results \label{sec:prevresults}}


% Location of the 2012--2014 Deviation, wavelet reconstruction
%\begin{wrapfigure}{r}{0.5\textwidth} 
%\vspace{-50pt}
%\includegraphics[width=\linewidth]{Figures/deviant.pdf}
%\caption[Location of the 2012--2014 Deviation]{The 2012--2014 deviation in Greenland mass and the total from the reconstructed modeled wavelet coefficients. By adding in the real values of only four wavelet coefficients back into the modeled wavelet reconstruction we improve the variance explanation by $15\%$. These wavelets are shown inset, weighted by their values in September 2012, the extreme of the deviation, and are concentrated in southwestern Greenland. ``m$_{\zeta\gamma}$'' refers to a wavelet basis function ``m'' of index $\gamma$ in level $\zeta$. This Figure appeared in my Spring JP.
%\label{fig:deviant}}
%\end{wrapfigure}

In my Spring 2018 JP, I explored the use of a 2-D wavelet basis to represent the GRACE gravimetric data over Greenland such that meaningfully contributing basis functions also contained information about spatial structure (see
Figure~\ref{fig:thegrid}). 

I developed a procedure for choosing the most important wavelet basis functions in order to extract the true fluctuation of the signal from the over-determined image calculated from the typical GRACE spherical harmonic basis. I then tested which wavelet basis function best captured the 2012--2013 deviation from the expected signal, finding the deviations to be concentrated in southwestern Greenland (see
Figure~\ref{fig:deviant}). 



\section{Data}

\subsection{Ice Data}

Observations of ice loss are measured remotely by weighing the mass by gravitational anomaly, photographing the surface area, and by measuring relative height of the ice with passive and active altimetry.

In addition to data sets which directly measure aspects of ice change, we have complimentary measurements which capture some of the processes which control melting, such as air temperature, humidity, and atmospheric pressure. Atmospheric data are measured directly at weather station all over the world, and are synthesized into large continuous data sets referred to as "reanalysis" data. "Reanalysis" data sets are the outputs of climate models, which use assumptions about the laws which govern atmospheric processes to compute simulations of the atmosphere constrained by directly observed weather data.
 
 \subsection{Atmospheric Data}


\section{Discussion}

%% using the NAO
Often times monthly means are reported when comparing an NAO index to a melt event \cite[][]{mattingly2018}, even though daily or sub-daily solutions are readily available, and despite the fact that the physical atmospheric drivers of melting (such as temperature, cloud cover, etc.) are changing on a daily, not monthly, basis.


%% is Greenland warming?
\subsection{Greenland is melting, is it warming?}
The melting of the Greenland Ice Sheet is strongly connected to warming temperatures in much of the literature, and the significant repercussions of various short-term warming scenarios on the evolution of ice sheet melting are of great importance to predicting sea-level rise over the next century \cite[][]{hahn2018,pattyn2018,ruckamp2018}. To best understand the relationship between warming temperatures and Greenland melt, some caution must be used in drawing a direct line between increasing mean global surface temperatures and increases of melt on the Greenland ice sheet. 

Between the late 1990's and 2012, an apparent pause in the rise of mean global temperatures coincided with the strongest acceleration in Greenland ice sheet mass loss of the last century \cite[][]{kosaka2013,angelen2014,bevis2018}. The ``hiatus'' observed in global warming over the first decade of the 21st century has been correlated with natural climate variability such as the El Ni\~{n}o/Southern Oscillation (ENSO), with heat being stored in cooler La Ni\~{n}a ocean waters \cite[][]{kosaka2013}. Between 2012 and 2017, mean global temperatures resumed a strong upwards trend driven by a negative to positive shift in ENSO phase (see Fig.~\ref{fig:ENSO}).

Over the same period, mean Greenland Ice Sheet near-surface temperature did not rise significantly, and in fact was decreasing slightly across almost the entire ice sheet between 2003 and 2017 \cite[see Figs.~\ref{fig:ENSO}~\&~\ref{fig:Tslope}, and ][]{hearty2018, westergaard2018,reeves2017}. Temperatures over Greenland are expected to continue to rise with global temperatures into the future, but have not done so significantly since the 1990's \cite[][]{reeves2017}


\begin{figure}[h]
\centering
\makebox[\textwidth][c]{\includegraphics[height=0.5\textheight]
{/Users/benjamingetraer/Documents/IndependentWork/senior-thesis/figures/spring2019/ENSO_TEMP.png}}
\caption[]{} \label{fig:ENSO}
\end{figure}

\begin{figure}[h]
\centering
\makebox[\textwidth][c]{\includegraphics[height=0.5\textheight]
{/Users/benjamingetraer/Documents/IndependentWork/senior-thesis/figures/spring2019/Tslope.png}}
\caption[]{} \label{fig:Tslope}
\end{figure}







\section{Next steps}

\begin{enumerate}
\item quantitative image to image comparison of moisture transport and mass loss
\item calculate mass loss on a sub basin spatial level across Greenland and compare to other estimates
\item pressure relationship
\end{enumerate}

The NAO index is unitless, and represents the relative magnitude of the 500mb pressure difference between Azores and Reykjav\'ik compared to the 1950--2000 monthly mean.

%http://www.cpc.ncep.noaa.gov/products/precip/CWlink/pna/nao_loading.html

% APPENDIX A --- DATA SOURCES
\newpage 
\appendix

\section{Appendix A: Data and code sources \label{app:a}}
\textit{RL05 spherical harmonic coefficients for the time-variant geopotential field from the GFZ, JPL, and CSR data processing centers are available at:} \\
\indent \url{ftp://podaac.jpl.nasa.gov/allData/grace/L2/}\\

\noindent\textit{Coefficients describing Earth's center of mass \cite[spherical harmonic degree~1, from][]{swenson2008} are available at:} \\
\indent\url{ftp://podaac-ftp.jpl.nasa.gov/GeodeticsGravity/tellus/L2/degree_1/}\\

\noindent\textit{Coefficients describing Earth's oblateness \cite[spherical harmonic degree~2, order~0, from][]{cheng2013} are available at:} \\
\indent \url{ftp://ftp.csr.utexas.edu/pub/slr/degree_2/}\\

\noindent\textit{Monthly values for the North Atlantic Oscillation Index are calculated by the Climate Prediction Center, with normalized monthly average values since January 1950 available at:} \\
\indent \url{ftp://ftp.cpc.ncep.noaa.gov/wd52dg/data/indices/nao_index.tim}\\

\noindent\textit{Reanalyzed MERRA-2 atmospheric data (3 dimensional, 6-hourly, instantaneous pressure-level analysis, V5.12.4) are calculated by NASA and made available by the Goddard Earth Sciences Data and Information Services Center at:} \\
\indent \url{https://goldsmr5.gesdisc.eosdis.nasa.gov/opendap/MERRA2/M2I6NPANA.5.12.4/}\\
Global Modeling and Assimilation Office (GMAO) (2015), MERRA-2 inst6_3d_ana_Np: 3d,6-Hourly,Instantaneous,Pressure-Level,Analysis,Analyzed Meteorological Fields V5.12.4, Greenbelt, MD, USA, Goddard Earth Sciences Data and Information Services Center (GES DISC), Accessed: [Data Access Date], 10.5067/A7S6XP56VZWS


\noindent\textit{MATLAB code for the expansion and manipulation of spherical harmonic
eigenfunctions into Slepian bases and manipulation of GRACE files is borrowed and adapted from:} \\
\indent \url{https://github.com/csdms-contrib/}\\

\noindent\textit{MATLAB code developed for this project, including functions for executing the wavelet analysis and scripts for generating figures, can be accessed at:} \\
\indent \url{https://github.com/bgetraer/slepian_bgetraer/}\\




%\noindent\textit{Outline coordinates for the Greenland ice scheet drainage basins from \cite{zwally} available at:}\\
%\indent \url{http://icesat4.gsfc.nasa.gov/cryo_data/ant_grn_drainage_systems.php}\\


%--References
\small
\renewcommand{\bibsep}{0em}

\renewcommand{\bibname}{References}
\bibliographystyle{/Users/benjamingetraer/Documents/IndependentWork/LatexFiles/gji.bst}
\bibliography{/Users/benjamingetraer/Documents/IndependentWork/LatexFiles/bgetraerBib.bib}




\end{document}